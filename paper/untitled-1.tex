\documentclass[citeauthoryear]{llncs} %
\usepackage[utf8]{inputenc}
\bibliographystyle{splncs}

\begin{document}

\title{Implementação de um robô reativo para o simulador Ciber-Rato}
\author{Tiago Babo e Hélder Moreira}

\institute{Faculdade de Engenharia da Universidade do Porto}

\maketitle

\begin{abstract}
This paper describes a reactive robotic agent for the Ciber-Rato platform. The algorithm used by the robot, based on the behaviour “follow wall” - when possible - doesn't save previous information and can finish some types of mazes easily.
\end{abstract}

\section{Introdução}
O simulador Ciber-Rato (\cite{microrato}) surgiu no âmbito de um concurso de robótica móvel organizado pela Universidade de Aveiro. O objetivo das equipas é construir um algoritmo de controlo que comanda um robô virtual, com a finalidade de resolver um labirinto. 

Para efeitos de desenvolvimento, foi considerado o cenário em que este tem de encontrar um objeto em específico (representado por um queijo, normalmente).

Este artigo descreve uma possível implementação de um robô reativo que consegue resolver um determinado número de configurações de obstáculos e labirintos, recorrendo a um algoritmo que tenta contornar paredes. 

\section{Robô Virtual}

O robô virtual é constituído por uma forma circular, equipado com sensores, atuadores e botões de comando. 

\subsection{Sensores}
Para o desenvolvimento do algoritmo foi considerado que o robô possui: três sensores de obstáculos, um sensor de farol, que lhe indica a direção da zona de chegada, um sensor de colisões binário e um sensor que lhe diz se está na área do objetivo. 

Os sensores têm diferentes resoluções, ruído associado e pode haver alguma latência na sua leitura.

\subsection{Atuadores}

O robô possui dois motores, representando duas rodas, que permitem efetuar translações ou rotações da sua posição. É possível especificar a força aplicada a cada roda. 

Quando o robô atinge o objetivo final deve sinalizar que lá chegou, através dos \emph{leds} que possui. 

\subsection{Botões de Comando}
Os botões de comando, \emph{Start} e \emph{Stop}, permitem ao simulador controlar o decorrer da competição, devendo o robô obedecer à sua alteração.

\section{Simulador}

O simulador Ciber-Rato é responsável por:
\begin{enumerate} 
\item Implementar o labirinto virtual 
\item Implementar o corpo virtual do robô
\item Coordenar o movimento dos robôs no labirinto em articulação com o agente
\item Desempenhar o papel de juiz virtual da prova:
\begin{enumerate} 
\item Controla o tempo da prova
\item Aplica as penalizações (choque com obstáculos, por exemplo)
\item Calcula as pontuações
\end{enumerate}
\end{enumerate}

O cenário da competição é delineado por uma área retangular, delimitada, composta por obstáculos, uma área objetivo e várias posições onde os robôs podem começar. Em relação aos obstáculos, estes podem ter alturas diferentes, influenciando a leitura do sensor de farol do robô. Se o obstáculo ultrapassar uma certa altura, o robô deixa de conseguir receber a informação proveniente do sensor. 

\section{Algoritmo usado}

\section{Conclusões}

\begingroup
\renewcommand\refname{Referências}
\begin{thebibliography}{}

\bibitem[Micro-Rato 2011]{microrato}
Micro-Rato, Departamento de Eletrónica e Telecomunicações, Universidade de Aveiro, 2011, "CiberRato 2011 - Rules and Technical Specifications" Acedido a 3 de Novembro 2012. http://microrato.ua.pt
\end{thebibliography}
\endgroup
\end{document}